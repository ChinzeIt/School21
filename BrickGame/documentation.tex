\documentclass[12pt]{article}

\usepackage{polyglossia}
\setdefaultlanguage{english}

\usepackage{geometry}
\geometry{margin=2.5cm}

\usepackage{listings}
\usepackage{xcolor}

\lstset{
    basicstyle=\ttfamily\footnotesize,
    backgroundcolor=\color{gray!10},
}

\title{Documentation for "BrickGame"}
\author{Alicashe}
\date{}

\begin{document}

\maketitle

\section{Introduction}

The project BrickGame started as a classic "Tetris" game (v1.0) implemented in C using the ncurses library for the terminal interface.  
It has been upgraded to BrickGame v2.0 with the addition of a classic "Snake" game implemented in C++ using the Qt framework for a desktop application.  
The game can be launched both as a desktop application and via a terminal interface.

\section{Project Structure}

\subsection{Tetris Library}

\begin{itemize}
    \item Backend source files are located in \texttt{brick\_game/tetris}.
\end{itemize}

\subsection{Snake Library}

\begin{itemize}
    \item Backend source files are in \texttt{brick\_game/snake}.
\end{itemize}

\subsection{Terminal Interface (\texttt{src/gui/cli})}

\begin{itemize}
    \item Handles rendering the game field using the ncurses library.
    \item Processes user input and displays the current game state.
\end{itemize}

\subsection{Desktop Interface (\texttt{src/gui/desktop})}

\begin{itemize}
    \item Uses the Qt framework for rendering.
    \item Handles visual display of the game field.
    \item Processes desktop UI button inputs.
\end{itemize}

\section{Building the Project}

The project uses the \texttt{make} build system with the following targets:

\begin{itemize}
    \item \texttt{all}: Build the project.
    \item \texttt{install}: Install the program to the system.
    \item \texttt{uninstall}: Remove the program from the system.
    \item \texttt{clean}: Remove temporary files and folders.
    \item \texttt{dvi}: Generate a DVI document.
    \item \texttt{dist}: Create an archive with the necessary files for building and running the program.
    \item \texttt{test}: Run unit tests.
    \item \texttt{lcov\_report}: Output test coverage in HTML format.
    \item \texttt{run\_game}: Launch the desktop application.
\end{itemize}

\section{Environment Requirements}

\begin{itemize}
    \item C11 and C++20 programming languages.
    \item g++ compiler.
    \item ncurses library for the terminal interface.
    \item Qt framework for the desktop interface.
\end{itemize}

\section{Installation and Running Instructions}

\begin{enumerate}
    \item \textbf{Installing dependencies:}
    \begin{itemize}
        \item Ensure g++ is installed.
        \item Install the ncurses library.
        \item Install the Qt framework.
        \item Install missing dependencies using:
        \begin{itemize}
            \item \texttt{sudo apt install g++}
            \item \texttt{sudo apt install libncurses5-dev}
            \item Qt installation may require sudo depending on your system.
        \end{itemize}
    \end{itemize}

    \item \textbf{Installation:}
    \begin{itemize}
        \item Run \texttt{make install} to install the program.
    \end{itemize}

    \item \textbf{Running:}
    \begin{itemize}
        \item Run \texttt{./game/CLI\_BrickGame 0} or \texttt{1} to start the desired game in terminal mode.
        \item Run \texttt{make run\_game} for the desktop application.
    \end{itemize}

    \item \textbf{Uninstallation:}
    \begin{itemize}
        \item Run \texttt{make uninstall} to remove the program.
    \end{itemize}
\end{enumerate}

\section{Using Tetris}

\subsection{Controls}
\begin{itemize}
    \item Left/Right arrow keys to move pieces horizontally.
    \item Down arrow key to speed up piece fall.
    \item Spacebar to rotate the piece.
    \item \texttt{p} to pause.
    \item \texttt{s} to resume from pause.
    \item \texttt{q} to quit.
\end{itemize}

\subsection{Game Mechanics}
\begin{itemize}
    \item Rotate and move pieces.
    \item Speed up falling pieces.
    \item Display the next piece.
    \item Clear completed lines.
    \item Game over when pieces reach the top.
\end{itemize}

\subsection{Game Over}
\begin{itemize}
    \item Game ends when no new piece can spawn.
    \item A "Game Over" menu is displayed.
\end{itemize}

\section{Using Snake}

\subsection{Controls}
\begin{itemize}
    \item The snake moves automatically every two seconds.
    \item Left and Right arrow keys control horizontal movement.
    \item Spacebar triggers a speed boost.
    \item \texttt{p} to pause the game.
    \item \texttt{s} to resume the game from pause.
    \item \texttt{q} to exit the game.
\end{itemize}

\subsection{Game Mechanics}
\begin{itemize}
    \item The snake moves continuously on its own.
    \item Eating an apple increases the snake's length.
    \item Only one apple is present on the field at any time.
    \item Each apple grants 1 point.
    \item Level increases every 5 points, up to a maximum of 10 levels.
    \item With each level, the snake's speed increases by 1 unit.
    \item Moving left or right changes the snake's direction accordingly.
\end{itemize}

\subsection{Collision and Game Over}
\begin{itemize}
    \item Collision with the walls or with itself ends the game.
    \item Victory is achieved when the player reaches 200 points.
    \item After game over, a "Game Over" menu is displayed.
\end{itemize}

\section{Testing}

The project includes unit tests:

\begin{itemize}
    \item C code is tested using \texttt{check} library (\texttt{test\_c}).
    \item C++ code is tested using \texttt{test\_cpp}.
    \item Test coverage is at least 80\%.
\end{itemize}

\end{document}
